%% start of file `template.tex'.
%% Copyright 2006-2010 Xavier Danaux (xdanaux@gmail.com).
%% Copyright 2010-2011 Mark Liu (markwayneliu@gmail.com).
%
% This work may be distributed and/or modified under the
% conditions of the LaTeX Project Public License version 1.3c,
% available at http://www.latex-project.org/lppl/.

\documentclass[10pt,letterpaper,sans]{moderncv}        % possible options include font size ('10pt', '11pt' and '12pt'), paper size ('a4paper', 'letterpaper', 'a5paper', 'legalpaper', 'executivepaper' and 'landscape') and font family ('sans' and 'roman')

% modern themes
\moderncvstyle{classic}                            % style options are 'casual' (default), 'classic', 'oldstyle' and 'banking'
\moderncvcolor{blue}                                % color options 'blue' (default), 'orange', 'green', 'red', 'purple', 'grey' and 'black'
%\renewcommand{\familydefault}{\sfdefault}         % to set the default font; use '\sfdefault' for the default sans serif font, '\rmdefault' for the default roman one, or any tex font name
%\nopagenumbers{}                                  % uncomment to suppress automatic page numbering for CVs longer than one page

% character encoding
\usepackage[utf8]{inputenc}                       % if you are not using xelatex ou lualatex, replace by the encoding you are using

% adjust the page margins
\usepackage[scale=0.84]{geometry}
%\setlength{\hintscolumnwidth}{3cm}                % if you want to change the width of the column with the dates
%\setlength{\makecvtitlenamewidth}{10cm}           % for the 'classic' style, if you want to force the width allocated to your name and avoid line breaks. be careful though, the length is normally calculated to avoid any overlap with your personal info; use this at your own typographical risks...

\usepackage{import}

% personal data
\firstname{Youngjin}
\familyname{Heo}
% \address{}
% \phone[mobile]{}                   % optional, remove / comment the line if not wanted
\email{heoyj@umich.edu}                               % optional, remove / comment the line if not wanted
\homepage{www.linkedin.com/in/youngjinheo}                         % optional, remove / comment the line if not wanted
%\extrainfo{additional information}                 % optional, remove / comment the line if not wanted


%----------------------------------------------------------------------------------
%            content
%----------------------------------------------------------------------------------
\begin{document}
%-----       resume       ---------------------------------------------------------

\makecvtitle
\vspace{-0.5cm}

\section{Education}
\vspace{1pt}

\cventry{Apr.2017}{MS, Biostatistics}{University of Michigan}{Ann Arbor, MI}{}{}

\cventry{Aug.2014}{MS, Mathematical Science}{Korea Advanced Institute of Science and Technology}{Daejeon, South Korea}{}{}

\cventry{Jan.2012}{BS, Mathematical Science}{Korea Advanced Institute of Science and Technology}{Daejeon, South Korea}{Minor in Financial Engineering Program}{}

\section{Work Experiences}
\vspace{1pt}

\cventry{Jun--Aug.2016}{Data Scientist Intern}{University of Michigan Transportation Research Institute}{Ann Arbor}{MI}
{
\begin{itemize}
\item Extracted hard-braking events in Michigan using a 5.2 TB dataset in Python using pandas, numpy and multiprocessing modules, and identified the duration and frequency of the events in order to avoid potential car damage.
\end{itemize}
\vspace{3pt}
}

\cventry{Jan--Jul.2015}{Statistician}{Clinical Trial Center in Chonbuk National Hospital}{Jeonju}{South Korea}
{
\begin{itemize}
\item Maintained data analysis pipeline for Phase I clinical trial and analyzed the trial datasets for writing 'statistical analysis plan (SAP)' and ‘clinical study report (CSR)’ using SAS, SQL and R in order to help pharmaceutical companies to make a decision on equivalence of two medicines or treatments.
\end{itemize}
}
\vspace{3pt}

\cventry{Sep.2013 -- Feb.2014}{Statistical Programmer}{University of California San Francisco}{San Francisco}{CA}
{
\begin{itemize}
\item Implemented supervised machine learning techniques in correlated structured data in R with glmnet package
to recognize informative patterns, select significant variables, and do classification for deriving clinically meaningful information from the data of patients with Alzheimer or Parkinson disease.
\end{itemize}
} 
\vspace{3pt}

\section{Academic Experiences}
\vspace{1pt}

\cventry{Jan--Apr.2017}{Statistical Modeling Projects}{Statistical Investigation course}{University of Michigan}{}
{
\begin{itemize}
\item Modeled and analyzed real world datasets using logistic regression, linear mixed model, survival analysis
and/or machine learning algorithms in R in order to figure out risk factors for disease-related outcomes or build predictive models for developing a disease.
\end{itemize}
}
\vspace{3pt}

\cventry{Jan--Apr.2017}{Signal Classification with Machine Learning Projects}{Signal Processing and Machine Learning course}{University of Michigan}{}
{
\begin{itemize}
\item Extracted features from signals and classified signals by using decision tree, Support Vector Machine
(SVM), and K-Nearest Neighbor (KNN), which resulted in improving accuracy of classification.
\item Applied Convolutional Neural Network (CNN) for biomedical image segmentation in order to classify cancers
 using Theano in Python.
\end{itemize}
}
\vspace{3pt}

\cventry{Sep--Dec.2016}{Block-wise Gibbs Sampling Project}{Statistical Computing course}{University of Michigan}{}
{
\begin{itemize}
\item Implemented the block-wise Gibbs Sampling in C++ for generating samples from a large dataset and
increased the sampling speed by 60 percent through modifying covariance computation algorithm so as to identify risk factors of the dataset.
\end{itemize}
}
\vspace{3pt}

\cventry{Sep.2016 -- Apr.2017}{Teaching Assistant}{Statistical Inference \& Probability and Distribution Theory}{University of Michigan}{}
{
\begin{itemize}
\item Assisted 80 graduate students by holding office hours to guide students to complete their assignments and understand materials in-depth, and graded their homeworks and exams.
\end{itemize}
}
\vspace{3pt}



\section{Technical Skills \& Language}
\vspace{1pt}

\cventry{\textbf{Programming}}{\mdseries Python, SQL, R, SAS, C++, Matlab, Linux, Hadoop, GitHub, Tableau}{}{}{}{}
\cventry{\textbf{Language}}{\mdseries English (Fluent in written and oral), Korean (Native)}{}{}{}{}

\section{Honor \& Awards}
\vspace{1pt}

\cvitemwithcomment{}{Korean Government Fellowship}{Sep. 2015 -- Apr.2017}
\cvitemwithcomment{}{Outstanding Teaching Assistant Award}{May. 2013, 2014}


\end{document}
%% end of file `template.tex'.



